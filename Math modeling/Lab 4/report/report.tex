\documentclass[a4paper]{article}

\usepackage[utf8]{inputenc}
\usepackage[T1,T2A]{fontenc}
\usepackage[english, ukrainian]{babel}

\usepackage{mathtext}
\usepackage{cmap}
\usepackage{amsmath}
\usepackage{amsfonts}
\usepackage{amsthm}
\usepackage{color}
\usepackage{tikz}
\usepackage{listings}

\setlength{\parskip}{0.5em}

\title{\vspace{-5em}Лабораторна робота №4 з \par
Математичного моделювання}
\author{Шкаліков Олег, ФІ-81}

\date{ 10 листопада 2020 р.}

\begin{document}

\maketitle

\usetikzlibrary{arrows,decorations.pathmorphing,backgrounds,positioning,fit,petri}

\section*{Завдання 1}
Побудувати граф досяжних розміток мережі Петрі.

\usetikzlibrary{arrows.meta}

\begin{center}
    \begin{tikzpicture}
    [node distance=1.3cm,on grid,>=stealth',bend angle=30,auto,
    every place/.style= {minimum size=6mm,thick,draw=blue!75,fill=blue!20},
    every transition/.style={thick,draw=black!75,fill=black!20},
    red place/.style= {place,draw=red!75,fill=red!20}]

    \node [place] at (-3, 2) (A) {1};
    \node [place] at (-3, 0) (B) {2};
    \node [place] at (3, 2) (C) {3};
    \node [place] at (3, 0) (D) {4};
    \node [place,tokens=1] at (0, -2) (E) {\hspace{0.3cm} 5};

    \node [transition] at (-6, 1) (t1) {A}
    edge [pre, bend right] (E)
    edge [post] (A)
    edge [post] (B);

    \node [transition] at (0, 2) (t2) {B}
    edge [pre] (A)
    edge [post] (C);

    \node [transition] at (0, 0) (t3) {C}
    edge [pre] (B)
    edge [post] (D);

    \node [transition] at (6, 1) (t4) {D}
    edge [pre] (C)
    edge [pre] (D)
    edge [post, bend left] (E)
    edge [post, bend left=10] (A);

    \end{tikzpicture}
\end{center}

\begin{center}
    \begin{tikzpicture}
        \begin{scope}[every node/.style={rectangle,thick,draw}]
            \node (A) at (-12,0) {0,0,0,0,1};
            \node (B) at (-8,0) {1,1,0,0,0};
            \node (C) at (-4,0) {0,0,1,1,0};
            \node (D) at (0, 0) {1,0,0,0,1};
            \node (E) at (0, -2.5) {1,1,n-1,0,0};
            \node (F) at (-4,-2.5) {0,0,n,1,0};
            \node (G) at (-8,-2.5) {1,0,n-1,0,1};
        \end{scope}

        \node (H) at (-12, -2.5) {\dots};

        \begin{scope}[>={Stealth[black]},
            every node/.style={fill=white,circle},
            every edge/.style={very thick, draw}]
            \path [->] (A) edge node{A} (B);
            \path [->] (B) edge node{B+C} (C);
            \path [->] (C) edge node{D} (D);
            \path [->] (D) edge node[align=center]{\dots \\ A} (E);
            \path [->] (E) edge node{B+C} (F);
            \path [->] (F) edge node{D} (G);
            \path [->] (G) edge node{A} (H);
        \end{scope}
    \end{tikzpicture}
\end{center}

Де $n$ - номер "ітерації" у циклі, який є у графі, що відповідає мережі
Петрі(кількість разів, коли перехід A був виконаним).

Як ми можемо бачити мережі Петрі не відповідає мова,
яка допускає скінченне слово(у графі досяжних розміток
нескінченність вершин та немає термінальних вершин).
\pagebreak
\section*{Завдання 2}
Для виконання завдання варіанту 4 було обрано метод
поступового ускладнення моделей з 1, 2 та 3 варіантів.
Тому розглянемо процес створення мережі Петрі покроково.
У доданках до цього протоколу наведено програмну реалізацію
описаних мереж Петрі на мові Python за допомогою яких ми перевіряли
корректність наведених моделей.
\subsection*{Варіант 1}

\begin{center}
    \begin{tikzpicture}
    [node distance=1.3cm,on grid,>=stealth',bend angle=25,auto,
    every place/.style= {minimum size=6mm,thick,draw=blue!75,fill=blue!20},
    every transition/.style={thick,draw=black!75,fill=black!20},
    red place/.style= {place,draw=red!75,fill=red!20}]

    \node [place,tokens=1] at (-6, 0) (A1) {};
    \node [place] at (-6, 3) (B1) {};
    \node [place,tokens=1] at (-7.5, 1) (C1) {};

    \node [place,tokens=1] at (-3, 0) (A2) {};
    \node [place] at (-3, 3) (B2) {};
    \node [place,tokens=1] at (-4.5, 1) (C2) {};

    \node [place,tokens=1] at (0, 0) (A3) {};
    \node [place] at (0, 3) (B3) {};
    \node [place,tokens=1] at (-1.5, 1) (C3) {};

    \node [place,tokens=1] at (3, 0) (A4) {};
    \node [place] at (3, 3) (B4) {};
    \node [place,tokens=1] at (1.5, 1) (C4) {};

    \node [place,tokens=1] at (6, 0) (A5) {};
    \node [place] at (6, 3) (B5) {};
    \node [place,tokens=1] at (4.5, 1) (C5) {};

    \node [transition] at (-6, 1.5) (f1) {}
    edge [pre] (A1)
    edge [pre, bend left] (C1)
    edge [pre, bend right] (C2)
    edge [post] (B1);

    \node [transition] at (-6, 4.5) (e1) {}
    edge [pre] (B1)
    edge [post] (C1)
    edge [post] (C2)
    edge [post, bend right] (A1);

    \node [transition] at (-3, 1.5) (f2) {}
    edge [pre] (A2)
    edge [pre, bend left] (C2)
    edge [pre, bend right] (C3)
    edge [post] (B2);

    \node [transition] at (-3, 4.5) (e2) {}
    edge [pre] (B2)
    edge [post] (C2)
    edge [post] (C3)
    edge [post, bend right] (A2);

    \node [transition] at (0, 1.5) (f3) {}
    edge [pre] (A3)
    edge [pre, bend left] (C3)
    edge [pre, bend right] (C4)
    edge [post] (B3);

    \node [transition] at (0, 4.5) (e3) {}
    edge [pre] (B3)
    edge [post] (C3)
    edge [post] (C4)
    edge [post, bend right] (A3);

    \node [transition] at (3, 1.5) (f4) {}
    edge [pre] (A4)
    edge [pre, bend left] (C4)
    edge [pre, bend right] (C5)
    edge [post] (B4);

    \node [transition] at (3, 4.5) (e4) {}
    edge [pre] (B4)
    edge [post] (C4)
    edge [post] (C5)
    edge [post, bend right] (A4);

    \node [transition] at (6, 1.5) (f5) {}
    edge [pre] (A5)
    edge [pre, bend left] (C5)
    edge [pre, bend left=50] (C1)
    edge [post] (B5);

    \node [transition] at (6, 4.5) (e5) {}
    edge [pre] (B5)
    edge [post] (C5)
    edge [post, bend right=25] (C1)
    edge [post, bend right] (A5);

    \end{tikzpicture}
\end{center}

Тут і надалі будемо нумерувати рівні позицій(кулі) та переходів(квадрати) знизу догори
(окремо позиції та переходи).
Тобто, вершини рівню 1 - це найнижчі 5 позицій у графі вище.
Позиції рівню 1 - це найнижчі 5 переходів у графі вище.

Опишемо значення переходів та позицій цього варіатну мережі Петрі.
Позиції:
\begin{enumerate}
    \item Філософ гуляє
    \item Виделка лежить на своєму місці
    \item Філософ обідає
\end{enumerate}

Переходи:
\begin{enumerate}
    \item Філософ бере 2 виделки
    \item Філософ повертається до прогулянки та кладе виделки на місце
\end{enumerate}

\subsection*{Варіант 2}


\begin{center}
    \begin{tikzpicture}
    [node distance=1.3cm,on grid,>=stealth',bend angle=25,auto,
    every place/.style= {minimum size=6mm,thick,draw=blue!75,fill=blue!20},
    every transition/.style={thick,draw=black!75,fill=black!20},
    red place/.style= {place,draw=red!75,fill=red!20}]

    \node [place,tokens=1] at (-6, 0) (A1) {};
    \node [place] at (-6, 3) (B1) {};
    \node [place,tokens=1] at (-7.5, 1) (C1) {};
    \node [place] at (-6, 6) (D1) {};

    \node [place,tokens=1] at (-3, 0) (A2) {};
    \node [place] at (-3, 3) (B2) {};
    \node [place,tokens=1] at (-4.5, 1) (C2) {};
    \node [place] at (-3, 6) (D2) {};

    \node [place,tokens=1] at (0, 0) (A3) {};
    \node [place] at (0, 3) (B3) {};
    \node [place,tokens=1] at (-1.5, 1) (C3) {};
    \node [place] at (0, 6) (D3) {};

    \node [place,tokens=1] at (3, 0) (A4) {};
    \node [place] at (3, 3) (B4) {};
    \node [place,tokens=1] at (1.5, 1) (C4) {};
    \node [place] at (3, 6) (D4) {};

    \node [place,tokens=1] at (6, 0) (A5) {};
    \node [place] at (6, 3) (B5) {};
    \node [place,tokens=1] at (4.5, 1) (C5) {};
    \node [place] at (6, 6) (D5) {};

    \node [transition] at (-6, 1.5) (f1) {}
    edge [pre] (A1)
    edge [pre, bend left] (C1)
    edge [post] (B1);

    \node [transition] at (-6, 4.5) (r1) {}
    edge [pre] (B1)
    edge [pre] (B2)
    edge [post] (A2)
    edge [post] (D1);

    \node [transition] at (-6, 7.5) (e1) {}
    edge [pre] (D1)
    edge [post] (C1)
    edge [post] (C2)
    edge [post, bend right] (A1);

    \node [transition] at (-3, 1.5) (f2) {}
    edge [pre] (A2)
    edge [pre, bend left] (C2)
    edge [post] (B2);

    \node [transition] at (-3, 4.5) (r2) {}
    edge [pre] (B2)
    edge [pre] (B3)
    edge [post] (A3)
    edge [post] (D2);

    \node [transition] at (-3, 7.5) (e2) {}
    edge [pre] (D2)
    edge [post] (C2)
    edge [post] (C3)
    edge [post, bend right] (A2);

    \node [transition] at (0, 1.5) (f3) {}
    edge [pre] (A3)
    edge [pre, bend left] (C3)
    edge [post] (B3);

    \node [transition] at (0, 4.5) (r3) {}
    edge [pre] (B3)
    edge [pre] (B4)
    edge [post] (A4)
    edge [post] (D3);

    \node [transition] at (0, 7.5) (e3) {}
    edge [pre] (D3)
    edge [post] (C3)
    edge [post] (C4)
    edge [post, bend right] (A3);

    \node [transition] at (3, 1.5) (f4) {}
    edge [pre] (A4)
    edge [pre, bend left] (C4)
    edge [post] (B4);

    \node [transition] at (3, 4.5) (r4) {}
    edge [pre] (B4)
    edge [pre] (B5)
    edge [post] (A5)
    edge [post] (D4);

    \node [transition] at (3, 7.5) (e4) {}
    edge [pre] (D4)
    edge [post] (C4)
    edge [post] (C5)
    edge [post, bend right] (A4);

    \node [transition] at (6, 1.5) (f5) {}
    edge [pre] (A5)
    edge [pre, bend left] (C5)
    edge [post] (B5);

    \node [transition] at (6, 4.5) (r3) {}
    edge [pre] (B5)
    edge [pre] (B1)
    edge [post] (A1)
    edge [post] (D5);

    \node [transition] at (6, 7.5) (e5) {}
    edge [pre] (D5)
    edge [post] (C5)
    edge [post, bend right=25] (C1)
    edge [post, bend right] (A5);

    \end{tikzpicture}
\end{center}

Опишемо значення переходів та позицій цього варіанту мережі Петрі.
Позиції:
\begin{enumerate}
    \item Філософ гуляє
    \item Виделка лежить на своєму місці
    \item Філософ взяв ліву виделку
    \item Філософ обідає
\end{enumerate}

Переходи:
\begin{enumerate}
    \item Філософ бере ліву виделки
    \item Філософ бере праву виделку(її передав йому правий сусід)
    \item Філософ повертається до прогулянки та кладе виделки на місце
\end{enumerate}

Як ми можемо бачити, граф мережі Петрі має багато дуг,
що значно ускладнює задачу його зображення на малюнку.
Проте помітимо, що граф нашой задачі можна розділити
на 5 блоків, що відповідають кожному з філосовів.
Тому для цього варіанту та надалі будемо малювати
малюнки лише для одного блоку.


\begin{center}
    \begin{tikzpicture}
    [node distance=1.3cm,on grid,>=stealth',bend angle=25,auto,
    every place/.style= {minimum size=6mm,thick,draw=blue!75,fill=blue!20},
    every transition/.style={thick,draw=black!75,fill=black!20},
    red place/.style= {place,draw=red!75,fill=red!20}]

    \node [place,tokens=1] at (0, 0) (A3) {};
    \node [place] at (0, 3) (B3) {};
    \node [place,tokens=1] at (-1.5, 1) (C3) {};
    \node [place] at (0, 6) (D3) {};

    \node [place,tokens=1] at (3, 0) (A4) {};
    \node [place] at (3, 3) (B4) {};
    \node [place,tokens=1] at (1.5, 1) (C4) {};

    \node [transition] at (0, 1.5) (f3) {}
    edge [pre] (A3)
    edge [pre, bend left] (C3)
    edge [post] (B3);

    \node [transition] at (0, 4.5) (r3) {}
    edge [pre] (B3)
    edge [pre] (B4)
    edge [post] (A4)
    edge [post] (D3);

    \node [transition] at (0, 7.5) (e3) {}
    edge [pre] (D3)
    edge [post] (C3)
    edge [post] (C4)
    edge [post, bend right] (A3);

    \node [transition] at (3, 1.5) (f4) {}
    edge [pre] (A4)
    edge [pre, bend left] (C4)
    edge [post] (B4);

    \end{tikzpicture}
\end{center}
\input{Tasks/2.3.tex}
\subsection*{Варіант 4}

Опишемо значення переходів та позицій цього варіанту мережі Петрі.
Позиції:
\begin{enumerate}
    \item Філософ гуляє
    \item Виделка лежить на своєму місці
    \item Філософ взяв ліву виделку
    \item Філософ взяв обидві виделки
    \item \begin{enumerate}
        \item (права гілка) Філософ обідає
        \item (середня гілка) Філософ не прийняв запрошення
        \item (ліва гілка) Філософ прийняв запрошення
    \end{enumerate}
\end{enumerate}

Переходи:
\begin{enumerate}
    \item Філософ бере ліву виделки
    \item Філософ бере праву виделку(її передав йому правий сусід)
    \item \begin{enumerate}
        \item (права гілка) Філософ готується до обіду
        \item (середня гілка) Філософ не погоджується на запрошення
        \item (ліва гілка) Філософ погоджується на запрошення
    \end{enumerate}
    \item Філософ повертається до прогулянки(ліва та права гілка).
    Виделка повертається на місце - середня гілка.
\end{enumerate}

\begin{center}
    \begin{tikzpicture}
    [node distance=1.3cm,on grid,>=stealth',bend angle=25,auto,
    every place/.style= {minimum size=6mm,thick,draw=blue!75,fill=blue!20},
    every transition/.style={thick,draw=black!75,fill=black!20},
    red place/.style= {place,draw=red!75,fill=red!20}]

    \node [place,tokens=1] at (-3, 0) (A2) {};
    \node [place] at (-3, 3) (B2) {};
    \node [place,tokens=1] at (-4.5, 1) (C2) {};

    \node [place,tokens=1] at (0, 0) (A3) {};
    \node [place] at (0, 3) (B3) {};
    \node [place,tokens=1] at (-1.5, 1) (C3) {};
    \node [place] at (0, 6) (E3) {};
    \node [place] at (2, 9) (D3) {};
    \node [place] at (-2, 9) (F3) {};
    \node [place] at (0, 9) (G3) {};

    \node [place,tokens=1] at (3, 0) (A4) {};
    \node [place] at (3, 3) (B4) {};
    \node [place,tokens=1] at (2, 1) (C4) {};

    \node [transition] at (0, 1.5) (f3) {}
    edge [pre] (A3)
    edge [pre, bend left] (C3)
    edge [post] (B3);

    \node [transition] at (0, 4.5) (r3) {}
    edge [pre] (B3)
    edge [pre] (B4)
    edge [post] (A4)
    edge [post] (E3);

    \node [transition] at (2, 10.5) (e3) {}
    edge [pre] (D3)
    edge [post, bend right=15] (C3)
    edge [post, bend left=15] (C4)
    edge [post, bend left] (A3);

    \node [transition] at (2, 7.5) (eat3) {}
    edge [pre] (E3)
    edge [post] (D3);

    \node [transition] at (-2, 7.5) (inv3) {}
    edge [pre, bend left=9] (A2)
    edge [pre] (B3)
    edge [post] (F3)
    edge [post, bend right=30] (A2);

    \node [transition] at (0, 8) (ninv3) {}
    edge [pre] (A2)
    edge [pre, bend left] (B3)
    edge [post] (G3)
    edge [post, bend right=10] (A2)
    edge [post, bend left] (A3);

    \node [transition] at (-2, 10.5) (einv3) {}
    edge [pre] (F3)
    edge [post, bend right] (C3)
    edge [post, bend right=25] (A3);

    \node [transition] at (0, 10.5) (eninv3) {}
    edge [pre] (G3)
    edge [post, bend right=5] (C3);

    \node [transition] at (3, 1.5) (f4) {}
    edge [pre] (A4)
    edge [pre, bend left] (C4)
    edge [post] (B4);

    \node [transition] at (-3, 1.5) (f2) {}
    edge [pre] (A2)
    edge [pre, bend left] (C2)
    edge [post] (B2);

    \end{tikzpicture}
\end{center}
\pagebreak
\section*{Додатки}
\definecolor{deepblue}{rgb}{0,0,0.5}
\definecolor{deepred}{rgb}{0.6,0,0}
\definecolor{deepgreen}{rgb}{0,0.5,0}

\lstset{
    breaklines=true,
    postbreak=\mbox{\textcolor{red}{$\hookrightarrow$}\space},
    otherkeywords={self},
    keywordstyle=\color{deepblue},
    emph={MyClass,__init__, @abstractmethod, @property},
    emphstyle=\color{deepred},
    stringstyle=\color{deepgreen},
    showstringspaces=false
}

Далі наведено програмний код імплементованих мереж Петрі.

\lstinputlisting[language=python]{../src/PetriNet.py}
\lstinputlisting[language=python]{../src/tasks.py}

\end{document}