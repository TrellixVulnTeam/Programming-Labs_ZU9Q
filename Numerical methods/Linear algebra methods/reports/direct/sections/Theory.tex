\section*{\centering Теоретичні відомості}
У даному комп'ютерному практикумі перед нами
постає задача знаходження розв'зку системи лінійних
алгебраїчних рівнянь, яку можна представити у вигляді:
\begin{equation} \label{eq:linear}
    Ax = b
\end{equation}
,де $A$ - матриця коефіцієнтів системи розмірності $n \times n$,
$x$ - невідомий вектор розмірності $n$,
$b$ - вектор правої частини розмірності $n$.

\noindent
Взагалі кажучи, існують декілька ідеї методів, для розв'язання
цієї задачі. Ми розглянемо так звані прямі методи, для знаходження
коренів. Якщо матриця $A$ системи \ref{eq:linear} є симетричною, то
можемо спробувати застосувати так званий метод квадратного кореня.
Він полягає у тому, щоб застосувати декомпозицію Холецького та представити
систему у наступному вигляді:
\begin{equation}
    Ax = LL^Tx=b
\end{equation}
де $L$ - нижньо-трикутна матриця. Отже, маємо:
\[
    \begin{pmatrix}
        a_{11} & \dots & a_{1n} \\
        \vdots & \ddots & \vdots \\
        a_{n1} & \dots  & a_{nn}
    \end{pmatrix} =
    \begin{pmatrix}
        a_{11} & &  \\
        \vdots & \ddots &  \\
        a_{n1} & \dots  & a_{nn}
    \end{pmatrix}
    \begin{pmatrix}
        a_{11} & \dots & a_{n1} \\
        & \ddots & \vdots \\
        & & a_{nn}
    \end{pmatrix}
\]
За правилами множення матриць, отримаємо:
\begin{equation} \label{alg:cholesky}
    \begin{gathered}
        l_{11} = \sqrt{a_{11}} \hspace{1cm}
        l_{i1} = \frac{a_{i1}}{l_{11}} \; (i > 1) \\
        l_{ij} = \frac{a_{ij} - \sum \limits_{k=1}^{i-1} l_{ik} l_{jk}}{l_{ij}} \; (1 < j \leq i)
    \end{gathered}
\end{equation}

\noindent
Маючи декомпозицію Холецького ми можемо розв'зати
систему \ref{eq:linear} у два кроки:
\[
    Ly = b \hspace{1cm} L^Tx = y
\]
Зазначимо, що матриці $L$ та $L^T$ - нижньо- та верхньо-трикутні відповідно,
тому їх можна розв'язати методами підстановки. Наведемо алгоритми цих методів.
\begin{algorithm}[h]
    \SetAlgoLined
    \KwIn{$A$, $b$}
    $x_1 \leftarrow \dfrac{b_1}{a_{11}} $ \\
    \For{$i \in \overline{2, n}$}
    {
        $x_i \leftarrow \dfrac{b_i - \sum \limits_{j=1}^{i-1} a_{ij}x_j}{a_{ii}}$
    }
    \KwOut{$x$}
    \caption{Пряма підстановка(для нижньо-трикутної матриці)}
    \label{alg:forward}
\end{algorithm}

\pagebreak
\begin{algorithm}[h]
    \SetAlgoLined
    \KwIn{$A$, $b$}
    $x_n \leftarrow \dfrac{b_n}{a_{nn}} $ \\
    \For{$i \in \{n-1, \dots, 1\}$}
    {
        $x_i \leftarrow \dfrac{b_i - \sum \limits_{j=i+1}^{n} a_{ij}x_{j}}{a_{ii}}$
    }
    \KwOut{$x$}
    \caption{Зворотна підстановка(для верхньо-трикутної матриці)}
    \label{alg:backward}
\end{algorithm}

\noindent
Інколи, застосування операції квадратного кореня значно ускладнене,
бо може призвести до значних чисельних похибок,
або підкореневий вираз є від'ємним(і потрібно переходити у комплексну площину).
Тому можна вдосконалити попередні алгоритми, та представити систему у вигляді:
\begin{equation}
    Ax = LDL^Tx=b
\end{equation}
де $L$ - нижньо-трикутна матриця з одиницями на діагоналі,
$D$ - діагональна матриця. За правилами множення матриць, отримаємо:
\begin{equation} \label{alg:ldl}
    \begin{gathered}
        d_{1} = a_{11} \hspace{1cm}
        l_{ii} = 1 \hspace{1cm}
        l_{i1} = \frac{a_{i1}}{d_{i}} \; (i>1) \\
        d_{i} = a_{ii} - \sum \limits_{k=1}^{i-1} l_{ik}^2 d_k \; (i>1) \hspace{1cm}
        l_{ij} = \dfrac{a_{ij} - \sum \limits_{k=1}^{i-1} l_{ik} l_{jk} d_k}{d_{j}} \; (1<j<i)
    \end{gathered}
\end{equation}
