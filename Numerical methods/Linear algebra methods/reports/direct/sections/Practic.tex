\section*{\centering Практична частина}

Розглянемо застосування описаних методів на прикладі наступної системи:

\[
    A = \begin{pmatrix}
        5.5 & 7 & 6 & 5.5 \\
        7 & 10.5 & 8 & 7 \\
        6 & 8 & 10.5 & 9 \\
        5.5 & 7 & 9 & 10.5
    \end{pmatrix} \hspace{1cm}
    b = \begin{pmatrix}
        23 \\
        32 \\
        33 \\
        31
    \end{pmatrix}
\]
Помітимо, що матриця $A$ цієї системи симметрична,
тому ми можемо спробувати застосувати декомпозицію
Холецького та LDL.

\subsection*{Декомпозиція Холецького}

У навчальних цілях наведемо результат роботи(заповнення матриці $L$)
алгоритму \ref{alg:cholesky} на кожній ітерації зовнішнього циклу.
Зазначимо, що "початкова" матриця нульова і кожне число округлене
до 6 значущих цифр.

\begin{gather*}
    L_1 = \begin{pmatrix}
        2.34521 & 0 & 0 & 0 \\
        0 & 0 & 0 & 0 \\
        0 & 0 & 0 & 0 \\
        0 & 0 & 0 & 0
    \end{pmatrix} \hspace{1cm}
    L_2 = \begin{pmatrix}
        2.34521 & 0 & 0 & 0 \\
        2.98481 & 1.26131 & 0 & 0 \\
        0 & 0 & 0 & 0 \\
        0 & 0 & 0 & 0
    \end{pmatrix} \\
    L_3 = \begin{pmatrix}
        2.34521 & 0 & 0 & 0 \\
        2.98481 & 1.26131 & 0 & 0 \\
        2.55841 & 0.2883 & 1.96759 & 0 \\
        0 & 0 & 0 & 0
    \end{pmatrix} \hspace{1cm}
    L_4 = \begin{pmatrix}
        2.34521 & 0 & 0 & 0 \\
        2.98481 & 1.26131 & 0 & 0 \\
        2.55841 & 0.2883 & 1.96759 & 0 \\
        2.34521 & 0 & 1.52471 & 1.63563
    \end{pmatrix}
\end{gather*}
Таким чином отримаємо:
\begin{gather*}
    Ax = LL^Tx= b \\
    Ly = b \hspace{1cm}
    L^T x = y
\end{gather*}
Тепер розв'яжемо системи за допомогою
алгоритмів \ref{alg:forward} та
\ref{alg:backward} відповідно(округлені до 16 значущої цифри).
\[
    y = \begin{pmatrix}
        9.807232952358079 \\
        2.162249910469345 \\
        3.702853335674256 \\
        1.43935204774342
    \end{pmatrix} \hspace{1cm}
    x = \begin{pmatrix}
        0.1599999999999985 \\
        1.440000000000001 \\
        1.199999999999999 \\
        0.8800000000000007
    \end{pmatrix}
\]
При обрахунку вектора нев'язки $r = Ax - b$,
ми отримали нульовий вектор, що пов'язуємо з тим,
що отриманий результат дуже близький до
точного кореня $(0.16, 1.44, 1.2, 0.88)$. Та усі розбіжності
нівелюються точність операції над типом \textit{double},
який має обжену кількість розвядів у двійковому представленні.

\subsection*{LDL декомпозиція}

Як і у попередньому прикладі наведемо результат роботи
алгоритму \ref{alg:ldl} на кожній ітерації зовнішнього циклу.

\begin{gather*}
    L_1 = \begin{pmatrix}
        1 & 0 & 0 & 0 \\
        0 & 0 & 0 & 0 \\
        0 & 0 & 0 & 0 \\
        0 & 0 & 0 & 0
    \end{pmatrix}
    d_1 = \begin{pmatrix}
        5.5 \\ 0 \\ 0 \\ 0
    \end{pmatrix} \hspace{1cm}
    L_2 = \begin{pmatrix}
        1 & 0 & 0 & 0 \\
        1.27273 & 1 & 0 & 0 \\
        0 & 0 & 0 & 0 \\
        0 & 0 & 0 & 0
    \end{pmatrix}
    d_2 = \begin{pmatrix}
        5.5 \\ 1.59091 \\ 0 \\ 0
    \end{pmatrix} \\
    L_3 = \begin{pmatrix}
        1 & 0 & 0 & 0 \\
        1.27273 & 1 & 0 & 0 \\
        1.09091 & 0.228571 & 1 & 0 \\
        0 & 0 & 0 & 0
    \end{pmatrix} \hspace{1cm}
    d_3 = \begin{pmatrix}
        5.5 \\ 1.59091 \\ 3.87143 \\ 0
    \end{pmatrix} \\
    L_4 = \begin{pmatrix}
        1 & 0 & 0 & 0 \\
        1.27273 & 1 & 0 & 0 \\
        1.09091 & 0.228571 & 1 & 0 \\
        1 & 0 & 0.774908 & 1
    \end{pmatrix}
    d_4 = \begin{pmatrix}
        5.5 \\ 1.59091 \\ 3.87143 \\ 2.67528
    \end{pmatrix}
\end{gather*}
Таким чином отримаємо:
\begin{gather*}
    Ax = LDL^Tx= b \\
    Ly = b \hspace{0.5cm}
    Dz = y \hspace{0.5cm}
    L^T x = z
\end{gather*}
Тепер розв'яжемо системи за допомогою
алгоритмів \ref{alg:forward} та
\ref{alg:backward}(округлені до 16 значущої цифри).
\[
    y = \begin{pmatrix}
        23 \\
        2.727272727272727 \\
        7.285714285714285 \\
        2.354243542435427
    \end{pmatrix} \hspace{1cm}
    z = \begin{pmatrix}
        4.181818181818182 \\
        1.714285714285713 \\
        1.881918819188191 \\
        0.8800000000000009
    \end{pmatrix} \hspace{1cm}
    x = \begin{pmatrix}
        0.1600000000000028 \\
        1.439999999999998 \\
        1.199999999999999 \\
        0.8800000000000009
    \end{pmatrix}
\]
При обрахунку вектора нев'язки $r = Ax - b$,
ми отримали нульовий вектор, що пов'язуємо з тим,
що отриманий результат дуже близький до
точного кореня $(0.16, 1.44, 1.2, 0.88)$. Та усі розбіжності
нівелюються точність операції над типом \textit{double},
який має обжену кількість розвядів у двійковому представленні.
