\section*{\centering Практична частина}

Розглянемо приклад використання описаних
вище методів для розв'зання задачі знаходження
дійсних коренів рівняння ($16 \mod{10}= 6$ ) варіанту:

\[
    P(x) = 2x^3 - 4x^2 - x + 2 = 0
\]
\[
    a_3 = 2, \; a_2 = -4, \; a_1 = -1, \; a_0 = 2
\]

\subsection*{Відділення коренів}

Перед усім помітимо, що за наслідками з основної теореми алгебри
та теореми Безу рівняння має 3 комплексні корені, причому,
якщо число з ненульовою уявною частиною є коренем, то спряжене до нього
також є коренем.
Далі застосуємо теореми, наведені у теоретичній частині для
визначення границь та проміжків у яких ми будемо шукати корені.

За теоремою про границі комплексних коренів(\ref{th:bounds}) маємо:
\[
    A = 4, B = 4, a_0 = 2, a_3 = 2
\]
\[
    \frac{2}{2+4} = \frac{1}{3} \leq |x| \leq 3 = \frac{2+4}{2}
\]

Помітимо, що задане рівняння не вдовольняє умові теореми Гюа(\ref{th:gua}),
тому ми не можемо стверджувати, що рівняння має пару комплексноспряжених коренів.
Проте, за властивостями імплікації(логічного наслідку)
ми ще не можемо стверджувати і обернене твердження, що
комплексноспряжених коренів немає.

За допомогою теореми про верхню межу додатніх коренів(\ref{th:upper_bound})
оцінимо проміжки у яких можуть знаходитись від'ємні та
додатні корені рівняння.

\begin{table}[h!]
    \centering
    \begin{tabular}{|c|c|c|c|c|}
        \hline
        & $x$ & $\dfrac{1}{y}$ & $-y$ & $-\dfrac{1}{y}$ \\
        \hline
        m & 2 & 2 & 1 & 1 \\
        \hline
        $B$ & 4 & 4 & 2 & 4 \\
        \hline
        $a_0$ & 2 & 2 & 2 & 2 \\
        \hline
        R & 3 & 3 & 2 & 2.4 \\
        \hline
    \end{tabular}
\end{table}

Таким чином маємо наступні оцінки:
\[
    -2 \leq x^{-} \leq -\frac{1}{2.4} < -0.4 \hspace{1cm}
    0.3 < \frac{1}{3} \leq x^{+} \leq 3
\]

Застосуємо спосіб Лагранжа для уточнення верхньої межі додатніх коренів.
Помітимо, що:

\begin{align*}
    f(x) &= F(x) + H(x) \\
    F(x) &= 2x^3 - 4x^2 - x \\
    H(x) &= 2
\end{align*}
Розглянемо точку $2.5$. Значення функції $F$ у цій точці дорівнює:
$F(2.5) = 3.75$. А отже, за наслідком \ref{th:lagrange} маємо, що
значення $2.5$ можна обрати за верхню межу додатніх коренів.

За теоремою Штурма(\ref{th:sturm}) обрахуємо кількість додатніх та
від'ємних коренів($N$-кількість змін знаків).
Для даного рівняння маємо наступну послідовність Штурма:

\begin{align*}
    p_0 &= 2x^3 - 4x^2 - x + 2 \\
    p_1 &= 6x^2 - 8x -1 \\
    p_2 &= \frac{22}{9}x - \frac{16}{9} \\
    p_3 &= \frac{142884}{39204}
\end{align*}

\begin{table}[h!]
    \centering
    \begin{tabular}{|c|c|c|c|c|c|}
        \hline
        $p$ & -2 & -0.4 & 0.3 & 2.5 \\
        \hline
        $p_0$ & - & + & + & + \\
        \hline
        $p_1$ & + & + & - & + \\
        \hline
        $p_2$ & - & - & - & + \\
        \hline
        $p_3$ & + & + & + & + \\
        \hline
        N & 3 & 2 & 2 & 0 \\
        \hline
    \end{tabular}
\end{table}

Бачимо, що саме на проміжку $[0.3, 2.5]$ наявні 2 корені, тому
оберемо деяку точку у цьому проміжку(наприклад $1.5$) та
проведемо обрахунки знову. Таким чином, отримаємо 3 проміжки: $[-2, -0.4]$, $[0.3, 1.5]$, $[1.5, 2.5]$ ,- у кожному
з яких є тільки один корінь вихідного рівняння.

\begin{table}[h!]
    \centering
    \begin{tabular}{|c|c|c|c|c|c|}
        \hline
        $p$ & -2 & -0.4 & 0.3 & 1.5 & 2.5 \\
        \hline
        $p_0$ & - & + & + & - & + \\
        \hline
        $p_1$ & + & + & - & + & + \\
        \hline
        $p_2$ & - & - & - & + & + \\
        \hline
        $p_3$ & + & + & + & + & + \\
        \hline
        N & 3 & 2 & 2 & 1 & 0 \\
        \hline
    \end{tabular}
\end{table}

\pagebreak
\subsection*{Уточнення коренів}

Для кожного з отриманих проміжків запустимо аглоритми,
описані у теоретичній частині для уточнення коренів. Для кожної ітерації
наведемо значення, що використовує критерій
зупинки та специфічні для кожного методу величини
(поточна область пошуку або поточна точка, тощо).
Результати округлені до 6 знаку після коми.

\subsubsection*{Метод бісекцій}
Для методу бісекцій(\ref{alg:bisection}) визначимо параметр критерія зупинки
$\varepsilon$ рівним $10^{-5}$.
Результати запусків наведені у таблицях нижче.

\begin{table}[h!]
    \centering
    \begin{tabular}{|c|c|c|c|c|}
        \hline
        \textbf{Номер ітерації} & $a$ & $b$ & $c$ & $|b-a|$ \\
        \hline
        0 & -2.0 & -0.4 & -1.2 & 1.6 \\
        \hline
        1 & -1.2 & -0.4 & -0.8 & 0.8 \\
        \hline
        2 & -0.8 & -0.4 & -0.6 & 0.4 \\
        \hline
        3 & -0.8 & -0.6 & -0.7 & 0.2 \\
        \hline
        4 & -0.8 & -0.7 & -0.75 & 0.1 \\
        \hline
        5 & -0.75 & -0.7 & -0.725 & 0.05 \\
        \hline
        6 & -0.725 & -0.7 & -0.7125 & 0.025 \\
        \hline
        7 & -0.7125 & -0.7 & -0.70625 & 0.0125 \\
        \hline
        8 & -0.7125 & -0.70625 & -0.709375 & 0.00625 \\
        \hline
        9 & -0.709375 & -0.70625 & -0.707813 & 0.003125 \\
        \hline
        10 & -0.707813 & -0.70625 & -0.707031 & 0.001563 \\
        \hline
        11 & -0.707813 & -0.707031 & -0.707422 & 0.000781 \\
        \hline
        12 & -0.707422 & -0.707031 & -0.707227 & 0.000391 \\
        \hline
        13 & -0.707227 & -0.707031 & -0.707129 & 0.000195 \\
        \hline
        14 & -0.707129 & -0.707031 & -0.70708 & 0.000098 \\
        \hline
        15 & -0.707129 & -0.70708 & -0.707104 & 0.000049 \\
        \hline
        16 & -0.707129 & -0.707104 & -0.707117 & 0.000024 \\
        \hline
        17 & -0.707117 & -0.707104 & -0.707111 & 0.000012 \\
        \hline
        18 & -0.707111 & -0.707104 & -0.707108 & 0.000006 \\
        \hline
    \end{tabular}
    \caption{Результати роботи метода бісекцій на відрізку $[-2, -0.4]$}
\end{table}

\begin{table}[h!]
    \centering
    \begin{tabular}{|c|c|c|c|c|}
        \hline
        \textbf{Номер ітерації} & $a$ & $b$ & $c$ & $|b-a|$ \\
        \hline
        0 & 0.3 & 1.5 & 0.9 & 1.2 \\
        \hline
        1 & 0.3 & 0.9 & 0.6 & 0.6 \\
        \hline
        2 & 0.6 & 0.9 & 0.75 & 0.3 \\
        \hline
        3 & 0.6 & 0.75 & 0.675 & 0.15 \\
        \hline
        4 & 0.675 & 0.75 & 0.7125 & 0.075 \\
        \hline
        5 & 0.675 & 0.7125 & 0.69375 & 0.0375 \\
        \hline
        6 & 0.69375 & 0.7125 & 0.703125 & 0.01875 \\
        \hline
        7 & 0.703125 & 0.7125 & 0.707812 & 0.009375 \\
        \hline
        8 & 0.703125 & 0.707812 & 0.705469 & 0.004687 \\
        \hline
        9 & 0.705469 & 0.707812 & 0.706641 & 0.002344 \\
        \hline
        10 & 0.706641 & 0.707812 & 0.707227 & 0.001172 \\
        \hline
        11 & 0.706641 & 0.707227 & 0.706934 & 0.000586 \\
        \hline
        12 & 0.706934 & 0.707227 & 0.70708 & 0.000293 \\
        \hline
        13 & 0.70708 & 0.707227 & 0.707153 & 0.000146 \\
        \hline
        14 & 0.70708 & 0.707153 & 0.707117 & 0.000073 \\
        \hline
        15 & 0.70708 & 0.707117 & 0.707098 & 0.000037 \\
        \hline
        16 & 0.707098 & 0.707117 & 0.707108 & 0.000018 \\
        \hline
        17 & 0.707098 & 0.707108 & 0.707103 & 0.000009 \\
        \hline
    \end{tabular}
    \caption{Результати роботи метода бісекцій на відрізку $[0.3, 1.5]$}
\end{table}

\begin{table}[h!]
    \centering
    \begin{tabular}{|c|c|c|c|c|}
        \hline
        \textbf{Номер ітерації} & $a$ & $b$ & $c$ & $|b-a|$ \\
        \hline
        0 & 1.5 & 2.5 & 2.0 & 1.0 \\
        \hline
        1 & 1.5 & 2.0 & 1.75 & 0.5 \\
        \hline
        2 & 1.75 & 2.0 & 1.875 & 0.25 \\
        \hline
        3 & 1.875 & 2.0 & 1.9375 & 0.125 \\
        \hline
        4 & 1.9375 & 2.0 & 1.96875 & 0.0625 \\
        \hline
        5 & 1.96875 & 2.0 & 1.984375 & 0.03125 \\
        \hline
        6 & 1.984375 & 2.0 & 1.992188 & 0.015625 \\
        \hline
        7 & 1.992188 & 2.0 & 1.996094 & 0.007812 \\
        \hline
        8 & 1.996094 & 2.0 & 1.998047 & 0.003906 \\
        \hline
        9 & 1.998047 & 2.0 & 1.999023 & 0.001953 \\
        \hline
        10 & 1.999023 & 2.0 & 1.999512 & 0.000977 \\
        \hline
        11 & 1.999512 & 2.0 & 1.999756 & 0.000488 \\
        \hline
        12 & 1.999756 & 2.0 & 1.999878 & 0.000244 \\
        \hline
        13 & 1.999878 & 2.0 & 1.999939 & 0.000122 \\
        \hline
        14 & 1.999939 & 2.0 & 1.999969 & 0.000061 \\
        \hline
        15 & 1.999969 & 2.0 & 1.999985 & 0.000031 \\
        \hline
        16 & 1.999985 & 2.0 & 1.999992 & 0.000015 \\
        \hline
        17 & 1.999992 & 2.0 & 1.999996 & 0.000008 \\
        \hline
    \end{tabular}
    \caption{Результати роботи метода бісекцій на відрізку $[1.5, 2.5]$}
\end{table}

\clearpage
\subsubsection*{Метод Ньютона}
Для методу Ньютона(\ref{alg:newton}) визначимо параметр критерія зупинки
$\varepsilon$ рівним $10^{-5}$. За початкову точку оберемо
лівий кінець визначених на попередньому кроці інтервалів.
Результати запусків наведені у таблицях нижче.

\begin{table}[h!]
    \centering
    \begin{tabular}{|c|c|c|}
        \hline
        \textbf{Номер ітерації} & $x$ & $|f(x)|$ \\
        \hline
        0 & -2.0 & 28.0 \\
        \hline
        1 & -1.282051 & 7.507072 \\
        \hline
        2 & -0.889388 & 1.681687 \\
        \hline
        3 & -0.734553 & 0.216403 \\
        \hline
        4 & -0.707882 & 0.005943 \\
        \hline
        5 & -0.707107 & 0.000005 \\
        \hline
    \end{tabular}
    \caption{Результати роботи метода Ньютона з початковою точкою $-2$}
\end{table}

\begin{table}[h!]
    \centering
    \begin{tabular}{|c|c|c|}
        \hline
        \textbf{Номер ітерації} & $x$ & $|f(x)|$ \\
        \hline
        0 & 0.3 & 1.394 \\
        \hline
        1 & 0.787413 & 0.291066 \\
        \hline
        2 & 0.706091 & 0.003716 \\
        \hline
        3 & 0.707107 & 0.000001 \\
        \hline
    \end{tabular}
    \caption{Результати роботи метода Ньютона з початковою точкою $0.3$}
\end{table}

\begin{table}[h!]
    \centering
    \begin{tabular}{|c|c|c|}
        \hline
        \textbf{Номер ітерації} & $x$ & $|f(x)|$ \\
        \hline
        0 & 1.5 & 1.75 \\
        \hline
        1 & 5.0 & 147.0 \\
        \hline
        2 & 3.651376 & 42.382727 \\
        \hline
        3 & 2.800049 & 11.745149 \\
        \hline
        4 & 2.303241 & 2.914097 \\
        \hline
        5 & 2.068301 & 0.516066 \\
        \hline
        6 & 2.004753 & 0.033449 \\
        \hline
        7 & 2.000026 & 0.000179 \\
        \hline
        8 & 2.0 & 0 \\
        \hline
    \end{tabular}
    \caption{fd}
    \caption{Результати роботи метода Ньютона з початковою точкою $1.5$}
\end{table}

\clearpage
\subsubsection*{Метод хорд}
Для методу хорд(\ref{alg:secant}) визначимо параметр критерія зупинки
$\varepsilon$ рівним $10^{-5}$.
Результати запусків наведені у таблицях нижче.

\begin{table}[h!]
    \centering
    \begin{tabular}{|c|c|c|c|c|}
        \hline
        \textbf{Номер ітерації} & $a$ & $b$ & $c$ & $|f(c)|$ \\
        \hline
        0 & -2.0 & -0.4 & -0.488121 & 1.302471 \\
        \hline
        1 & -2.0 & -0.488121 & -0.555323 & 0.979285 \\
        \hline
        2 & -2.0 & -0.555323 & -0.604142 & 0.703182 \\
        \hline
        3 & -2.0 & -0.604142 & -0.638338 & 0.48822 \\
        \hline
        4 & -2.0 & -0.638338 & -0.661674 & 0.331046 \\
        \hline
        5 & -2.0 & -0.661674 & -0.677312 & 0.220868 \\
        \hline
        6 & -2.0 & -0.677312 & -0.687664 & 0.145768 \\
        \hline
        7 & -2.0 & -0.687664 & -0.694461 & 0.095514 \\
        \hline
        8 & -2.0 & -0.694461 & -0.698899 & 0.062291 \\
        \hline
        9 & -2.0 & -0.698899 & -0.701787 & 0.040498 \\
        \hline
        10 & -2.0 & -0.701787 & -0.703662 & 0.026277 \\
        \hline
        11 & -2.0 & -0.703662 & -0.704878 & 0.017027 \\
        \hline
        12 & -2.0 & -0.704878 & -0.705665 & 0.011024 \\
        \hline
        13 & -2.0 & -0.705665 & -0.706174 & 0.007134 \\
        \hline
        14 & -2.0 & -0.706174 & -0.706504 & 0.004615 \\
        \hline
        15 & -2.0 & -0.706504 & -0.706717 & 0.002984 \\
        \hline
        16 & -2.0 & -0.706717 & -0.706855 & 0.00193 \\
        \hline
        17 & -2.0 & -0.706855 & -0.706944 & 0.001248 \\
        \hline
        18 & -2.0 & -0.706944 & -0.707001 & 0.000807 \\
        \hline
        19 & -2.0 & -0.707001 & -0.707039 & 0.000522 \\
        \hline
        20 & -2.0 & -0.707039 & -0.707063 & 0.000337 \\
        \hline
        21 & -2.0 & -0.707063 & -0.707078 & 0.000218 \\
        \hline
        22 & -2.0 & -0.707078 & -0.707088 & 0.000141 \\
        \hline
        23 & -2.0 & -0.707088 & -0.707095 & 0.000091 \\
        \hline
        24 & -2.0 & -0.707095 & -0.707099 & 0.000059 \\
        \hline
        25 & -2.0 & -0.707099 & -0.707102 & 0.000038 \\
        \hline
        26 & -2.0 & -0.707102 & -0.707104 & 0.000025 \\
        \hline
        27 & -2.0 & -0.707104 & -0.707105 & 0.000016 \\
        \hline
        28 & -2.0 & -0.707105 & -0.707105 & 0.000013 \\
        \hline
        29 & -2.0 & -0.707105 & -0.707106 & 0.000007 \\
        \hline
    \end{tabular}
    \caption{Результати роботи метода хорд на відрізку $[-2, -0.4]$}
\end{table}

\begin{table}[h!]
    \centering
    \begin{tabular}{|c|c|c|c|c|}
        \hline
        \textbf{Номер ітерації} & $a$ & $b$ & $c$ & $|f(c)|$ \\
        \hline
        0 & 0.3 & 1.5 & 0.832061 & 0.449249 \\
        \hline
        1 & 0.3 & 0.832061 & 0.702384 & 0.017277 \\
        \hline
        2 & 0.702384 & 0.832061 & 0.707186 & 0.00029 \\
        \hline
        3 & 0.702384 & 0.707186 & 0.707107 & 0.000001 \\
        \hline
    \end{tabular}
    \caption{Результати роботи метода хорд на відрізку $[0.3, 1.5]$}
\end{table}

\begin{table}[h!]
    \centering
    \begin{tabular}{|c|c|c|c|c|}
        \hline
        \textbf{Номер ітерації} & $a$ & $b$ & $c$ & $|f(c)|$ \\
        \hline
        0 & 1.5 & 2.5 & 1.733333 & 1.335704 \\
        \hline
        1 & 1.733333 & 2.5 & 1.877855 & 0.739303 \\
        \hline
        2 & 1.877855 & 2.5 & 1.948734 & 0.338106 \\
        \hline
        3 & 1.948734 & 2.5 & 1.979349 & 0.141164 \\
        \hline
        4 & 1.979349 & 2.5 & 1.991825 & 0.056694 \\
        \hline
        5 & 1.991825 & 2.5 & 1.996786 & 0.022414 \\
        \hline
        6 & 1.996786 & 2.5 & 1.99874 & 0.008806 \\
        \hline
        7 & 1.99874 & 2.5 & 1.999507 & 0.003451 \\
        \hline
        8 & 1.999507 & 2.5 & 1.999807 & 0.001351 \\
        \hline
        9 & 1.999807 & 2.5 & 1.999924 & 0.000529 \\
        \hline
        10 & 1.999924 & 2.5 & 1.99997 & 0.000207 \\
        \hline
        11 & 1.99997 & 2.5 & 1.999988 & 0.000081 \\
        \hline
        12 & 1.999988 & 2.5 & 1.999995 & 0.000032\\
        \hline
        13 & 1.999995 & 2.5 & 1.999998 & 0.000012 \\
        \hline
        14 & 1.999998 & 2.5 & 1.999999 & 0.000005 \\
        \hline
    \end{tabular}
    \caption{Результати роботи метода хорд на відрізку $[1.5, 2.5]$}
\end{table}

\pagebreak
\subsection*{Порівняння методів}

Наведемо порівняльну таблицю результатів запусків
досліджуваних методів для нашого рівняння.

\begin{table}[h!]
    \centering
    \begin{tabular}{|c|c|c|c|c|}
        \hline
        \textbf{Метод} & \textbf{Значення кореня} & \textbf{Кількість ітерацій} \\
        \hline
        Бісекцій & -0.707108 & 18 \\
        \hline
        Ньютона & -0.707107 & 5 \\
        \hline
        Хорд & -0.707106 & 29 \\
        \hline
        \hline
        Бісекцій & 0.707103 & 17 \\
        \hline
        Ньютона & 0.707107 & 3 \\
        \hline
        Хорд & 0.707107 & 3 \\
        \hline
        \hline
        Бісекцій & 1.999996 & 17 \\
        \hline
        Ньютона & 2 & 8 \\
        \hline
        Хорд & 1.999999 & 14 \\
        \hline
    \end{tabular}
    \caption{Порівняльна таблиця досліджуваних методів}
\end{table}
