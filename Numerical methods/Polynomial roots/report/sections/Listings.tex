\section*{\centering Додатки}

\definecolor{deepblue}{rgb}{0,0,0.5}
\definecolor{deepred}{rgb}{0.6,0,0}
\definecolor{deepgreen}{rgb}{0,0.5,0}

\lstset{
    breaklines=true,
    postbreak=\mbox{\textcolor{red}{$\hookrightarrow$}\space},
    otherkeywords={self},
    keywordstyle=\color{deepblue},
    % emph={MyClass,__init__, @abstractmethod, @property},
    emphstyle=\color{deepred},
    stringstyle=\color{deepgreen},
    showstringspaces=false
}

Далі наведено програмний код імплементованих алгоритмів.
Вихідний код, який було створено для даного практикума
(у тому числі \LaTeX),
у повному обсязі можна знайти за наступним
\href{https://colab.research.google.com/drive/139J6V_hXOcZL_K_-qSM1CSk3T0zTmQM9?usp=sharing}{посиланням}:

\subsubsection*{Метод бісекцій}
\lstinputlisting[language=python]{../code/bisection.py}

\subsubsection*{Метод Ньютона}
\lstinputlisting[language=python]{../code/newton.py}

\subsubsection*{Метод хорд}
\lstinputlisting[language=python]{../code/secant.py}